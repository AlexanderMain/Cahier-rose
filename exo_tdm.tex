\documentclass[12pt,a4paper]{exprog}

\usepackage[francais]{babel}
\usepackage[utf8]{inputenc}
\usepackage[T1]{fontenc}
\usepackage{amssymb,amsmath}
\usepackage{multicol,array}
\usepackage{graphicx,color,tikz}
\usepackage{tabularx}
\usepackage{array}
\usepackage{times,mathptmx}

%%%%%%%%%%%%%%%%%%%%%%%%%%%%%%%%%%%%%%%%%%%%%%%%%%%%%%%%%%%%%%%%%%%%%%%%
\makeatletter

\newcommand{\IR}{{\mathbb R}}
\newcommand{\IC}{\mathbb{C}}
\newcommand{\IQ}{\mathbf{Q}}
\newcommand{\C}{{\mathcal C}}
\newcommand{\U}{\mathcal{U}}
\newcommand{\I}{\mathbf{i}}
\newcommand{\IZ}{\mathbb{Z}}
\newcommand{\IN}{\mathbf{N}}
\newcommand{\Th}{\text{Th}}

\DeclareMathOperator{\e}{e}
\DeclareMathOperator{\sh}{sh}

\@ifundefined{leqslant}{}{\let\le=\leqslant}
\@ifundefined{geqslant}{}{\let\ge=\geqslant}

%% Intervals
\newcommand{\intervalcc}[1]{\mathopen[#1\mathclose]}
\newcommand{\intervaloo}[1]{\mathopen]#1\mathclose[}
\newcommand{\intervaloc}[1]{\mathopen]#1\mathclose]}
\newcommand{\intervalco}[1]{\mathopen[#1\mathclose[}
\newcommand{\bigintervalcc}[1]{\bigl[#1\bigr]}

% Vrai/Faux
\newcommand\drawbox{%
  \hbox{\vrule \vtop{\vbox{\hrule
        \hbox{\vrule width 0pt height 1.7ex depth 0.3ex\hskip 2ex}}%
      \hrule}\vrule}}

\newcounter{VF}

\newenvironment{VF}{\bgroup
  \def\VF@label{\textsf{(\alph{VF})} Vrai~: \drawbox\quad Faux~: \drawbox}%
  \begin{VF@gen}
  }{%
  \end{VF@gen}
  \egroup}

\newenvironment{VF*}{\bgroup
  \def\VF@label{\textsf{(\alph{VF})} \drawbox}%
  \begin{VF@gen}
  }{%
  \end{VF@gen}
  \egroup}

\newenvironment{VF@gen}{%
  \bgroup
  \ifnum\@itemdepth >\thr@@ \@toodeep   \else
    \advance\@itemdepth\@ne
    \setcounter{VF}{0}%
    \expandafter\list{\addtocounter{VF}{1}%
      \gdef\@currentlabel{\alph{VF}}%
      \VF@label}{%
      \settowidth{\labelwidth}{\VF@label}%
      \addtolength{\labelwidth}{6pt}%
      \setlength{\labelsep}{1em}%
      \setlength{\leftmargin}{\labelwidth}%
      \addtolength{\leftmargin}{\labelsep}%
      \setlength{\itemindent}{0pt}%
      \setlength{\topsep}{10.0pt plus 4.0pt minus 6.0pt}%
      \setlength{\itemsep}{5.0pt plus 2.5pt minus 1.0pt}%
      \def\makelabel##1{\hss \llap{##1}}%
    }%
    \fi
}{%
  \global\advance\@listdepth\m@ne \endtrivlist
  \egroup
}

\newenvironment{answer}{%
%  \setbox0=\vbox  % comment to see the answers
  \bgroup
  \par
  \color[RGB]{150,25,150}%
}{%
  \egroup
}

\renewcommand{\phi}{\varphi}
\makeatother
%%%%%%%%%%%%%%%%%%%%%%%%%%%%%%%%%%%%%%%%%%%%%%%%%%%%%%%%%%%%%%%%%%%%%%%
\course{Théorie des modèles}
\title{Exercices}

\begin{document}
\maketitle
\newcommand{\str}[1]{\mathcal{#1}}
\newcommand{\Ll}{\mathcal{L}}


\begin{question}
Soient $\str{M}$ une $\Ll$-structure, $A$ un sous-ensemble de $M$ et $a\in M$. On
dit que $a$ est \emph{algébrique} sur $A$ s'il existe une formule
$\phi(x,y_{1},\ldots,y_{n})$ et $a_{1},\ldots,a_{n}\in A$ tels que
$\str{M}\models\phi(a,a_{1},\ldots,a_{n})$ et l'ensemble $\{b\in
M|\str{M}\models\phi(b,a_{1},\ldots,a_{n})\}$ est fini. On note $\text{acl}^{\str{M}}(A)$
l'ensemble des éléments de $M$ algébriques sur $A$.
\begin{enumerate}
  \item $A\subset\text{acl}^{\str{M}}(A)$;
  \item si $a\in\text{acl}^{\str{M}}(A)$, alors
    $a\in\text{acl}^{\str{M}}(A_{0})$ pour
    un certain $A_{0}\subset A$ fini;
  \item si $A\subset B$, alors $\text{acl}^{\str{M}}(A)\subset\text{acl}^{\str{M}}(B)$;
  \item  $\text{acl}^{\str{M}}(\text{acl}^{\str{M}}(A))=\text{acl}^{\str{M}}(A)$.
  \item Supposons que $A\neq\emptyset$. L'ensemble $\text{acl}^{\str{M}}(A)$ est-il le
    domaine d'une sous-structure de $\str{M}$?
  \item Si $\str{M}\prec\str{N}$, alors
    $\text{acl}^{\str{M}}(A)=\text{acl}^{\str{N}}(A)$.
  \item Soit $\str{K}$ un corps algébriquement clos et $A\subset K$. Que vaut $\text{acl}^{\str{K}}(A)$?
\end{enumerate}
\end{question}
\begin{question}
  Soit $\str{M}$ une $\Ll$-structure et $A\subset M$. Soit $D\subset M^{n}$
  définissable sur $A$. Alors $\sigma(D)=D$ pour tout $\sigma\in\text{Aut}_{A}(\str{M})$.
\end{question}
\begin{question}
  Soit $\Ll=\{0,1,+,\cdot,<\}$, $\str{N}=(\mathbf{N},0,1,+,\cdot,<)$ et
  $T=\text{Th}(\str{N})$. Alors il existe $\str{M}\succ\str{N}$ et $a\in M$ tels que
  $\mathbf{N}<a$ et $a$ est divisible par $n$ pour tout $n\in\mathbf{N}$.
\end{question}
\begin{question}
  $\str{M}\prec\str{N}$ si et seulement si pour tout $\bar{a}\in M$, $\text{tp}^{\str{M}}(\bar{a})=\text{tp}^{\str{N}}(\bar{a})$.
\end{question}
\begin{question}
  Soit $\str{M}$ une $\Ll$-structure et $A\subset M$.
  \begin{enumerate}
  \item Soit $D\subset M^{n+1}$ définissable sur $A$. Alors la projection de $D$
    sur $M^{n}$ est définissable sur $A$.
  \item On dit d'une fonction $f:M^{\ell}\to M^{k}$ qu'elle est
    \emph{définissable} sur $A$ si son graphe est un sous-ensemble de
    $M^{\ell+k}$ définissable sur $A$.
    \begin{enumerate}
    \item Soit $\str{K}$ un corps. Montrez que l'application
      $\text{det}:K^{2\times 2}\to K$ est définissable.
    \item Donnez d'autres exemples de fonctions définissables.
    \end{enumerate}
  \end{enumerate}
\end{question}
\begin{question}
  Soit $\str{M}$ une $\Ll$-structure et posons $T=\text{Th}(\str{M})$. Soient
  $c_{1},\ldots,c_{n}$ des constantes qui ne sont pas dans $\Ll$. Soit $T'$ une
  $\Ll\cup\{c_{1},\cdots,c_{n}\}$-théorie complète contenant $T$ et notons
  $p(x_{1},\ldots,x_{n})$ l'ensemble de $\Ll$-formules
$\{\phi(x_{1},\ldots,x_{n})|\phi(c_{1},\ldots,c_{n})\in T'\}$. Alors, il existe
$\str{N}\models T$ et $\bar{a}\in N$ tels que $\text{tp}^{\str{N}}(\bar{a})=p(x_{1},\ldots,x_{n})$.
\end{question}
\begin{question}
  Soit $\Ll=\{+,-,0,\cdot q|q\in\mathbf{Q}\}$, où $\cdot q$ est un symbole de
  fonction unaire. Soit $T_{\mathbf{Q}}$ la $\Ll$-théorie des
  $\mathbf{Q}$-espaces vectoriels.
  \begin{enumerate}
  \item Donnez une axiomatisation de $T_{\mathbf{Q}}$;
  \item montrez que, dans $T_{\mathbf{Q}}$, toute formule existentielle est
    équivalente à une formule sans quantificateurs;
  \item en déduire que $T_{\mathbf{Q}}$ a l'élimination des quantificateurs;
  \item montrez que $\mathbf{R}$, vu comme $\mathbf{Q}$-espace vectoriel, est
    l'union d'une chaîne élémentaire de $\mathbf{Q}$-sous-espaces vectoriels
    propres.
  \end{enumerate}
\end{question}

\begin{question}
  En admettant que $ACF$ a l'élimination des quantificateurs, montrez que
  $(\mathbf{C},+,-,\cdot,0,1)$ est l'union d'une chaîne propre de corps
  algébriquement clos.
\end{question}

\begin{question}
  Soit $\Ll=\{E\}$, où $E$ est un symbole de relation binaire.
  \begin{enumerate}
  \item Soit $T_{1}$ la théorie qui exprime la propriété suivante: $E$ est une
    relation d'équivalence ayant, pour chaque $n\in\mathbf{N}^{>0}$, une unique
    classe d'équivalence contenant $n$ éléments.
    \begin{enumerate}
    \item Axiomatisez $T_{1}$;
    \item $T_{1}$ est-elle $\aleph_{0}$-catégorique?
    \item montrez que $|S_{1}(T)|\ge\aleph_{0}$;
    \item Montrez que $T_{1}$ n'a pas l'élimination des quantificateurs;
    \item Soit $\Ll'=\Ll\cup\{c_{n}|n\in\IN^{>0}\}$. Soit $T'_{1}$ la
      $\Ll'$-théorie contenant $T$ exprimant le fait que $c_{n}$ est en relation
      avec exactement $n$ éléments, pour chaque $n\in\mathbf{N}$. Montrez que
      $T'_{1}$ a l'élimination des quantificateurs.
    \end{enumerate}
  \item Soit $T_{2}$ la théorie qui exprime la propriété suivante: $E$ est une
    relation d'équivalence ayant une infinite de classes d'équivalence, toutes
    infinies.
    \begin{enumerate}
    \item $T_{2}$ est-elle $\aleph_{0}$-catégorique?
    \item $T$ a-t-elle l'élimination des quantificateurs?
    \end{enumerate}
  \end{enumerate}
\end{question}

\begin{question}
  Soit $\Ll=\{<\}$ et notons $T=\text{DLO}$ la théorie des ordres totaux denses
  sans extrémité.
  \begin{enumerate}
  \item Axiomatisez $T$;
  \item Décrire les formules sans quantificateur, à équivalence modulo $T$;
  \item Montrez que $T$ a l'élimination des quantificateurs;
  \item $T$ est-elle complète?
  \item Soit $\str{M}$ une $\Ll$-structure et soit $A\subset M$. Soient $U,L$
    deux sous-ensembles de $A$. On dira que $(L,U)$ est une coupure de $A$ si
    $A=L\cup U$, $L\cap
    U=\emptyset$ et $L<U$, c'est-à-dire que pour tout $a\in L$ et tout $b\in U$,
    $a<b$\footnote{Remarquez que l'on autorise $L=\emptyset$ ou $U=\emptyset$
      dans la définition de coupure.}.
    Supposons que $\str{M}\models T$. Soit $p\in S_{1}(A)$ et notons
    $L_{p}$ (resp. $U_{p}$) l'ensemble
    $\{a|a<x\in p\}$ (resp. $\{b|\neg(b<x)\in p)\}$).
    \begin{enumerate}
    \item Montrez que si $p\in S_{1}(A)$ est réalisé par un élément de $A$,
      alors $p$ est isolé;
    \item Montrez que le couple
      $(L_{p},U_{p})$ est une coupure;
    \item Soient $p$ et $q$ deux types sur $A$, non réalisés dans $A$. Montrez
      que $p\neq q$ implique $(L_{p},U_{p})\neq (L_{q},U_{q})$;
    \item Etant donnée une coupure $(L,U)$ de $A$, l'ensemble
      \[\{a<x|a\in L\}\cup\{\neg(b<x)|b\in U\}\]
      détermine-t-il un type non réalisé dans $A$?
    \end{enumerate}
  \item Supposons que $\str{M}=(\IQ,<)$ et soit $A\subset \IQ$.
    \begin{enumerate}
    \item Quels sont les types à paramètres dans $A$ qui sont
      isolés\footnote{Distinguez en fonction de l'existence d'extrema dans
        $L_{p}$ et $U_{p}$.}?
    \item Donnez des exemples de types omis par $\str{M}$;
    \item Calculez $|S_{1}(\IQ)|$.
    \end{enumerate}
  \end{enumerate}
\end{question}
\end{document}
%%% Local Variables:
%%% mode: latex
%%% TeX-master: t
%%% End:
