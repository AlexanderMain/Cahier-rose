\documentclass[12pt,a4paper]{exprog}

\usepackage[francais]{babel}
\usepackage[utf8]{inputenc}
\usepackage[T1]{fontenc}
\usepackage{amssymb,amsmath, amsthm}
\usepackage{multicol,array}
\usepackage{graphicx,color,tikz}
\usepackage{tabularx}
\usepackage{array}
\usepackage{times,mathptmx}

%%%%%%%%%%%%%%%%%%%%%%%%%%%%%%%%%%%%%%%%%%%%%%%%%%%%%%%%%%%%%%%%%%%%%%%%
\makeatletter

\newcommand{\IR}{{\mathbb R}}
\newcommand{\IC}{\mathbb{C}}
\newcommand{\C}{{\mathcal C}}
\newcommand{\U}{\mathcal{U}}
\newcommand{\I}{\mathbf{i}}
\newcommand{\IZ}{\mathbb{Z}}
\newcommand{\IN}{\mathbf{N}}
\newcommand{\Th}{\text{Th}}

\DeclareMathOperator{\e}{e}
\DeclareMathOperator{\sh}{sh}

\@ifundefined{leqslant}{}{\let\le=\leqslant}
\@ifundefined{geqslant}{}{\let\ge=\geqslant}

%% Intervals
\newcommand{\intervalcc}[1]{\mathopen[#1\mathclose]}
\newcommand{\intervaloo}[1]{\mathopen]#1\mathclose[}
\newcommand{\intervaloc}[1]{\mathopen]#1\mathclose]}
\newcommand{\intervalco}[1]{\mathopen[#1\mathclose[}
\newcommand{\bigintervalcc}[1]{\bigl[#1\bigr]}

% Vrai/Faux
\newcommand\drawbox{%
  \hbox{\vrule \vtop{\vbox{\hrule
        \hbox{\vrule width 0pt height 1.7ex depth 0.3ex\hskip 2ex}}%
      \hrule}\vrule}}

\newcounter{VF}

\newenvironment{VF}{\bgroup
  \def\VF@label{\textsf{(\alph{VF})} Vrai~: \drawbox\quad Faux~: \drawbox}%
  \begin{VF@gen}
  }{%
  \end{VF@gen}
  \egroup}

\newenvironment{VF*}{\bgroup
  \def\VF@label{\textsf{(\alph{VF})} \drawbox}%
  \begin{VF@gen}
  }{%
  \end{VF@gen}
  \egroup}

\newenvironment{VF@gen}{%
  \bgroup
  \ifnum\@itemdepth >\thr@@ \@toodeep   \else
    \advance\@itemdepth\@ne
    \setcounter{VF}{0}%
    \expandafter\list{\addtocounter{VF}{1}%
      \gdef\@currentlabel{\alph{VF}}%
      \VF@label}{%
      \settowidth{\labelwidth}{\VF@label}%
      \addtolength{\labelwidth}{6pt}%
      \setlength{\labelsep}{1em}%
      \setlength{\leftmargin}{\labelwidth}%
      \addtolength{\leftmargin}{\labelsep}%
      \setlength{\itemindent}{0pt}%
      \setlength{\topsep}{10.0pt plus 4.0pt minus 6.0pt}%
      \setlength{\itemsep}{5.0pt plus 2.5pt minus 1.0pt}%
      \def\makelabel##1{\hss \llap{##1}}%
    }%
    \fi
}{%
  \global\advance\@listdepth\m@ne \endtrivlist
  \egroup
}

\newenvironment{answer}{%
%  \setbox0=\vbox  % comment to see the answers
  \bgroup
  \par
  \color[RGB]{150,25,150}%
}{%
  \egroup
}

\renewcommand{\phi}{\varphi}
\makeatother
%%%%%%%%%%%%%%%%%%%%%%%%%%%%%%%%%%%%%%%%%%%%%%%%%%%%%%%%%%%%%%%%%%%%%%%


\usepackage[colorlinks, linkcolor=blue]{hyperref}

\title{Cahier rose}
\course{Théorie des modèles I}

\theoremstyle{definition} \newtheorem{thm}{Th\'{e}or\`{e}me}
\theoremstyle{definition} \newtheorem{quest}{Question}
\theoremstyle{definition} \newtheorem{dev}[quest]{Devoir}

\newcommand{\tp}[2]{\mathrm{tp}^{\mathcal{#1}}\left(#2\right)}

\begin{document}
\maketitle

\begin{quest}
  Donner une axiomatisation de la théorie des groupes abéliens
  sans torsion, divisibles dans le langage $\{+, -, 0\}$. Cette
  théorie est-elle $\aleph_0$-catégorique?
\end{quest}

\begin{quest}
  Soient $\mathcal M$, $\mathcal N$ deux $\mathcal L$-structures,
  $\bar{a}\in M^n$ et $\bar{b}\in N^n$.

  Montrer que $\tp{M}{\bar{a}} = \tp{N}{\bar{b}}$ est équivalent
  à $f: M\to N: \bar{a}\mapsto \bar{b}$ est une application
  partielle élémentaire.
\end{quest}

\begin{quest}
  Soit $\mathcal{L}$ un langage et $\mathcal{M}$ une $\mathcal L$-%
  structure.
  \begin{enumerate}
  \item Définir l'expression \og $\mathcal M$ est $\kappa$-saturée\fg{},
    où $\kappa$ est un cardinal.
  \item Existe-t-il une $\mathcal L$-structure infinie $\mathcal N$
    $|N|^+$-saturée (où si $\kappa$ est un cardinal, $\kappa^+$ désigne
    le successeur de $\kappa$)?

  \end{enumerate}
\end{quest}

\begin{dev}
  Soit $\mathcal L = \{ +, -, 0, \cdot q, q\in\mathbb Q\}$ où
  $\cdot q$ est un symbole de fonction unaire.

  Soit $T_{\mathbb Q}$ la $\mathcal L$-théorie des $\mathbb Q$-espaces
  vectoriels non triviaux.
  \begin{enumerate}
  \item Donner une axiomatisation de $T_{\mathbb Q}$.
  \item Montrer que $T_{\mathbb Q}$ admet l'élimination
    des quantificateurs (dans le langage $\mathcal L$).
    \begin{enumerate}
    \item Montrer que dans $T_{\mathbb Q}$, toute formule
      existentielle est équivalente à une formule sans quantificateur.
    \item En déduire que $T_{\mathbb Q}$ a l'éliminatation
      des quantificateurs.
    \end{enumerate}
  \item Montrer que $\mathbb R$ vu comme $\mathbb Q$-espace
    vectoriel est l'union d'une chaîne élémentaire de
    $\mathbb Q$-sous-espaces vectoriels propres.
  \end{enumerate}
\end{dev}

\begin{quest}
  Soit $\mathcal L=\{+, -, \cdot, 0, 1\}$.
  Soit $K$ un corps commutatif vu comme $\mathcal L$-structure.
  Soit $M_2(K)$ l'anneau des matrices $2\times 2$ à coefficients
  dans $K$.

  Montrer que le groupe des matrices inversibles de $M_2(K)$
  est un sous-ensemble définissable de $K^4$, modulo
  l'identification suivante:
  \begin{equation*}
    \left( \begin{matrix}
        a_{1 1} & a_{1 2} \\
        a_{2 1} & a_{2 2}
      \end{matrix} \right)\in M_2(K)
    \rightarrow (a_{1 1}, a_{1 2}, a_{2 1}, a_{2 2}) \in K^4
  \end{equation*}
\end{quest}

\begin{quest}
  \'{E}noncer le théorème de Lowenheim-Skolem descendant.
\end{quest}

\begin{dev}
  Soit $E$ un symbole de relation binaire $\mathcal L= \{E\}$.
  \'{E}crire une $\mathcal L$-théorie qui exprime que $E$ est
  une relation d'équivalence avec pour chaque naturel $n\geq 1$
  une seule classe d'équivalence contenant exactement $n$ éléments.
  \begin{enumerate}
  \item $T$ est-elle $\aleph_0$ catégorique?
  \item Montrer que $|S_1(T)|\geq \aleph_0$.
  \item Comme $\mathcal L$ ne contient pas de constante,
    on dira que $T$ a l'élimination des quantificateurs
    si pour tout $\mathcal L$-formule $\varphi(x_1, \ldots, x_n)$,
    il existe une formule sans quantificateur
    $\psi(x_1, \ldots, x_{n+1})$ telle que
    $$T\models \forall x_1 \ldots \forall x_{n+1}
    (\varphi\leftrightarrow \psi).$$
    Est-ce que $T$ a l'élimination des quantificateurs? Justifiez
    votre réponse.
  \end{enumerate}
\end{dev}

\begin{quest}
  \'{E}noncer le théorème de Vaught.
\end{quest}

\begin{dev}
  Soit $\mathcal{L}= \{+, -, \cdot, 0, 1\}$ et
  $\mathcal{L}_{<} = \mathcal{L}\cup \{<\}$.
  \begin{enumerate}
  \item\sloppypar Montrer que dans la théorie des anneaux ordonnés, toute
    formule sans quantificateur $\psi(x_1, \ldots, x_n)$ peut
    se mettre sous la forme
    \begin{equation}
      \bigvee\limits_i \left(
        \bigwedge\limits_j p_j(\bar{x}) > 0 \land
        \bigwedge\limits_k q_k(\bar{x})=0
      \right)
    \end{equation}
    avec $p_j(x_1, \ldots, x_n)$,
    $q_k(x_1, \ldots, x_n)\in\mathbb Z[x_1, \ldots, x_n]$
  \item Montrer que $(\mathbb R, +, -, \cdot, 0, 1, <)$ n'a pas
    l'élimination des quantificateurs dans le langage $\mathcal L$.
  \end{enumerate}

\end{dev}

\begin{thm}[Tarski]
  $(\mathbb R, +, -, \cdot, 0, 1, <)$ a l'élimination des
  quantificateurs. dans le langage $\mathcal L_{<}$.
\end{thm}

\begin{quest}
  \'{E}noncer la définition de modèle atomique.
\end{quest}

\end{document}

%%% Local Variables:
%%% mode: latex
%%% TeX-master: t
%%% End:
