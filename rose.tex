\documentclass[a4paper, 12pt]{article}

\usepackage[utf8]{inputenc}
\usepackage[T1]{fontenc}
\usepackage[french]{babel}
\usepackage{amsmath, amssymb, amsthm, verbatim}
\usepackage{IEEEtrantools}
\usepackage[margin=1in]{geometry}
\usepackage[colorlinks, linkcolor=blue]{hyperref}
\usepackage{epigraph}
\usepackage{mathrsfs}
\usepackage{tikz-cd}
\usepackage{xcolor}
\usepackage{dsfont}


\newcommand{\ind}{\mathds{1}}
\newcommand{\ssi}{si et seulement si}
\newcommand{\st}{\ensuremath \mathrm{st}}
\renewcommand{\Re}{\mathrm{Re}}
\renewcommand{\Im}{\mathrm{Im}}

\newcommand{\lstar}[1]{\ensuremath \!\!\!\phantom{i}^*#1}
\newcommand{\rstar}{\lstar\mathbb{R}}
\title{Cahier rose}
\author{Théorie des modèles I}
\date{\today}


\theoremstyle{definition} \newtheorem{quest}{Question}
\theoremstyle{definition} \newtheorem{dev}[quest]{Devoir}

\newcommand{\tp}[2]{\mathrm{tp}^{\mathscr{#1}}\left(#2\right)}

\begin{document}
\maketitle

\begin{quest}
  Donner une axiomatisation de la théorie des groupes abéliens
  sans torsion, divisibles dans le langage $\{+, -, 0\}$. Cette
  théorie est-elle $\aleph_0$-catégorique?
\end{quest}

\begin{quest}
  Soient $\mathscr M$, $\mathscr N$ deux $\mathscr L$-structures,
  $\bar{a}\in M^n$ et $\bar{b}\in N^n$.

  Montrer que $\tp{M}{\bar{a}} = \tp{N}{\bar{b}}$ est équivalent
  à $f: M\to N: \bar{a}\mapsto \bar{b}$ est une application
  partielle élémentaire.
\end{quest}

\begin{quest}
  Soit $\mathscr{L}$ un langage et $\mathscr{M}$ une $\mathscr L$-%
  structure.
  \begin{enumerate}
  \item Définir l'expression \og $\mathscr M$ est $\kappa$-saturée\fg{},
    où $\kappa$ est un cardinal.
  \item Existe-t-il une $\mathscr L$-structure infinie $\mathscr N$
    $|N|^+$-saturée (où si $\kappa$ est un cardinal, $\kappa^+$ désigne
    le successeur de $\kappa$)?

  \end{enumerate}
\end{quest}

\begin{dev}
  Soit $\mathscr L = \{ +, -, 0, \cdot q, q\in\mathbb Q\}$ où
  $\cdot q$ est un symbole de fonction unaire.

  Soit $T_{\mathbb Q}$ la $\mathscr L$-théorie des $\mathbb Q$-espaces
  vectoriels non triviaux.
  \begin{enumerate}
  \item Donner une axiomatisation de $T_{\mathbb Q}$.
  \item Montrer que $T_{\mathbb Q}$ admet l'élimination
    des quantificateurs (dans le langage $\mathscr L$).
    \begin{enumerate}
    \item Montrer que dans $T_{\mathbb Q}$, toute formule
      existentielle est équivalente à une formule sans quantificateur.
    \item En déduire que $T_{\mathbb Q}$ a l'éliminatation
      des quantificateurs.
    \end{enumerate}
  \item Montrer que $\mathbb R$ vu comme $\mathbb Q$-espace
    vectoriel est l'union d'une chaîne élémentaire de
    $\mathbb Q$-sous-espaces vectoriels propres.
  \end{enumerate}
\end{dev}

\begin{quest}
  Soit $\mathscr L=\{+, -, \cdot, 0, 1\}$.
  Soit $K$ un corps commutatif vu comme $\mathscr L$-structure.
  Soit $M_2(K)$ l'anneau des matrices $2\times 2$ à coefficients
  dans $K$.

  Montrer que le groupe des matrices inversibles de $M_2(K)$
  est un sous-ensemble définissable de $K^4$, modulo
  l'identification suivante:
  \begin{equation*}
    \left( \begin{matrix}
        a_{1 1} & a_{1 2} \\
        a_{2 1} & a_{2 2}
      \end{matrix} \right)\in M_2(K)
    \rightarrow (a_{1 1}, a_{1 2}, a_{2 1}, a_{2 2}) \in K^4
  \end{equation*}
\end{quest}

\begin{quest}
  \'{E}noncer le théorème de Lowenheim-Skolem descendant.
\end{quest}

\begin{dev}
  Soit $E$ un symbole de relation binaire $\mathscr L= \{E\}$.
  \'{E}crire une $\mathscr L$-théorie qui exprime que $E$ est
  une relation d'équivalence avec pour chaque naturel $n\geq 1$
  une seule classe d'équivalence contenant exactement $n$ éléments.
  \begin{enumerate}
  \item $T$ est-elle $\aleph_0$ catégorique?
  \item Montrer que $|S_1(T)|\geq \aleph_0$.
  \item Comme $\mathscr L$ ne contient pas de constante,
    on dira que $T$ a l'élimination des quantificateurs
    si pour tout $\mathscr L$-formule $\varphi(x_1, \ldots, x_n)$,
    il existe une formule sans quantificateur
    $\psi(x_1, \ldots, x_{n+1})$ telle que
    $$T\models \forall x_1 \ldots \forall x_{n+1}
    (\varphi\leftrightarrow \psi).$$
    Est-ce que $T$ a l'élimination des quantificateurs? Justifiez
    votre réponse.
  \end{enumerate}
\end{dev}

\begin{quest}
  \'{E}noncer le théorème de Vaught.
\end{quest}

\begin{dev}
  Soit $\mathscr{L}= \{+, -, \cdot, 0, 1\}$ et
  $\mathscr{L}_{<} = \mathscr{L}\cup \{<\}$.
  \begin{enumerate}
  \item Montrer que dans la théorie des anneaux ordonnés, toute
    formule sans quantificateur $\psi(x_1, \ldots, x_n)$ peut
    se mettre sous la forme
    \begin{equation}
      \bigvee\limits_i \left (\bigwedge\limits_j p_j(\bar{x}) > 0 \land \bigwedge\limits_k q_k(\bar{x})=0\right)
    \end{equation}
    avec $p_j(x_1, \ldots, x_n)$,
    $q_k(x_1, \ldots, x_n)\in\mathbb Z[x_1, \ldots, x_n]$
  \item Montrer que $(\mathbb R, +, -, \cdot, 0, 1, <)$ n'a pas
    l'élimination des quantificateurs dans le langage $\mathscr L$.
  \end{enumerate}

\end{dev}

% \begin{thm}[Tarski]
%   $(\mathbb R, +, -, \cdot, 0, 1, <)$ a l'eq. dans le langage $\mathscr L_{<}$.
% \end{thm}


\end{document}

%%% Local Variables:
%%% mode: latex
%%% TeX-master: t
%%% End:
